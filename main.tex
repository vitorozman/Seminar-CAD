\documentclass[a4paper,12pt]{article}
\usepackage[slovene]{babel}
\usepackage[utf8]{inputenc}
\usepackage[T1]{fontenc}
\usepackage{lmodern}
\usepackage{amsmath,amsfonts}
\usepackage{enumitem}
\pagestyle{empty}
\begin{document}

\newcommand{\N}{\mathbb{N}}
\newcommand{\R}{\mathbb{R}}


\title{Diskretne Coonsove ploskveva urnikov}
\author{Matej Rojec, Vito Rozman}
\date{\today}

\maketitle

\tableofcontents

\section{Uvod in motivacija}

\subsection{Uvod v Bézierjeve ploskeve}
Bézierjevo ploskev $\mathbf{p} : [0,1]^2 \rightarrow \R^3$ iz tenzorskega produkta stopnje $(m, n) \in \N \times \N$  
definiramo s parametrizacijo:
$$\mathbf{p}(u,v) := \sum_{i=0}^m \sum_{j=0}^n \mathbf{b}_{i,j} B_i^m(u)B_j^n(v),$$
kjer sta $u,v$ iz enotskega kvadrata, t.j $(u,v) \in [0,1]^2$ ter $(\frac{i}{m}, \frac{j}{n})$
domenske točke, ki ustrezajo kontrolni točki $\mathbf{b}_{i,j}$.

Pri fiksnem $v$, množica $\{\mathbf{b} (u,v) \mid u \in [0,1]  \}$ predstavlja 
kontrolnimi točkami    $$\sum_{j=0}^n \mathbf{b}_{i,j} B_j^n(v), \qquad i=0,1,\ldots,m$$

ki so izračunane kot točke na Bézierjevih krivuljah stopnje $n$ pri parametru $v$.


\subsection{Coonsove ploskve}

Denimo, da imamo podane kontrole točke $\mathbf{b}_{i,j},~ i=0,\ldots,m,~ j=0,\ldots,n$.
Te določajo štiri robne krivulje 
$$\mathbf{x}(u,0),\qquad \mathbf{x}(u,1),\qquad \mathbf{x}(0,1),\qquad \mathbf{x}(1,v),$$

kjer je domena enotski kvadrat, t.j.~$(u,v) \in [0,1]^2$.
Te omejujejo iskano ploskev $\mathbf{p}$ iz tenzorskega produkta stopnje $(n,m)$.
Kontrolne točke

% tole je treba še mal popravt
\begin{align*}
      &\mathbf{b}_{0,0} &\mathbf{b}_{0,1} & &\ldots & &\mathbf{b}_{0,m-1} & &\mathbf{b}_{0,m} \\
      &\mathbf{b}_{1,0}  &  & &  & &  & &\mathbf{b}_{1,m} \\
      &\vdots  &  &  & &  & & &\vdots\\
      &\mathbf{b}_{n-1,0}  &  &  & &  & &  &\mathbf{b}_{n-1,m-1} \\ \\
      &\mathbf{b}_{n,0} &\mathbf{b}_{n,1} & &\ldots & &\mathbf{b}_{n,m-1} & &\mathbf{b}_{n,m} \\
\end{align*}

\end{document}