\documentclass[a4paper,12pt]{article}
\usepackage[slovene]{babel}
\usepackage[utf8]{inputenc}
\usepackage[T1]{fontenc}
\usepackage{lmodern}
\usepackage{amsmath,amsfonts}
\usepackage{enumitem}
\pagestyle{empty}
\begin{document}

\newcommand{\N}{\mathbb{N}}
\newcommand{\R}{\mathbb{R}}


\title{Diskretne Coonsove ploskveva urnikov}
\author{Matej Rojec, Vito Rozman}
\date{\today}

\maketitle

\tableofcontents

\section{Uvod in motivacija}

\subsection{Uvod v Bézierjeve ploskeve}
Bézierjevo ploskev $\mathbf{p} : [0,1]^2 \rightarrow \R^3$ iz tenzorskega produkta stopnje $(m, n) \in \N \times \N$  
definiramo s parametrizacijo:
$$\mathbf{p}(u,v) := \sum_{i=0}^m \sum_{j=0}^n \mathbf{b}_{i,j} B_i^m(u)B_j^n(v),$$
kjer sta $u,v$ iz enotskega kvadrata, t.j $(u,v) \in [0,1]^2$ ter $(\frac{i}{m}, \frac{j}{n})$
domenske točke, ki ustrezajo kontrolni točki $\mathbf{b}_{i,j}$.

Pri fiksnem $v$, množica $\{\mathbf{b} (u,v) \mid u \in [0,1]  \}$ predstavlja 
kontrolnimi točkami
$$\sum_{j=0}^n \mathbf{b}_{i,j} B_j^n(v), \qquad i=0,1,\ldots,m,$$
ki so izračunane kot točke na Bézierjevih krivuljah stopnje $n$ pri parametru $v$.


\subsection{Coonsove ploskve}

Denimo, da imamo podane kontrole točke $\mathbf{b}_{i,j},~ i=0,\ldots,m,~ j=0,\ldots,n$.
Te določajo štiri robne krivulje 
Te omejujejo iskano ploskev $\mathbf{p}$ iz tenzorskega produkta stopnje $(n,m)$.
Kontrolne točke
\begin{align*}
      &\mathbf{b}_{0,0} &\mathbf{b}_{1,0} & &\ldots & &\mathbf{b}_{m-1,0} & &\mathbf{b}_{m,0} \\
      &\mathbf{b}_{0,1}  &  & &  & &  & &\mathbf{b}_{m,1} \\
      &\vdots  &  &  & &  & & &\vdots\\
      &\mathbf{b}_{0,n-1}  &  &  & &  & &  &\mathbf{b}_{m,n-1} \\ 
      &\mathbf{b}_{0,n} &\mathbf{b}_{1,n} & &\ldots & &\mathbf{b}_{m-1,n} & &\mathbf{b}_{m,n} \\
\end{align*}
določajo štiri Bézierjev krivulje: 
\begin{align*}
   &\mathbf{p}(u,0) =\sum_{i=0}^m \mathbf{b}_{i,0} B_i^n(u),  \\
   &\mathbf{p}(u,1) =\sum_{i=0}^m \mathbf{b}_{i,n} B_i^n(u),  \\
   &\mathbf{p}(0,v) =\sum_{j=0}^n \mathbf{b}_{0,j} B_j^n(v),  \\
   &\mathbf{p}(1,v) =\sum_{j=0}^n \mathbf{b}_{m,j} B_j^n(v),  \\
\end{align*}
kjer je domena enotski kvadrat, t.j.~$(u,v) \in [0,1]^2$.
Kontrolne točke torej omejujejo ploskev $\mathbf{p}$. 
Sedaj potrebujemo definirati še ostale kontrolne točke 
$\mathbf{b}_{i,j},~ i=1,\ldots, m-1~, j=1,\ldots,n-1$. 
V ta namen definiramo tri dodatne ploskve:
\begin{enumerate}
   \item Prva je Bézierjeva ploskev stopnje $(m, 1)$, ki je kot ploskev stopnje $(m,n)$ podana s kontrolnimi točkami:
   $$\mathbf{b}_{i,j}^{(1)} = \left(1-\frac{j}{n}  \right) \mathbf{b}_{i,0} + \frac{j}{n} \mathbf{b}_{i,n} $$
   \item Druga je Bézierjeva ploskev stopnje $(1, n)$, ki je kot ploskev stopnje $(m,n)$ podana s kontrolnimi točkami:
   $$\mathbf{b}_{i,j}^{(2)} = \left(1-\frac{i}{m}  \right) \mathbf{b}_{0,j} + \frac{i}{m} \mathbf{b}_{m,j} $$
   \item Tretja je Bézierjeva ploskev stopnje $(1, 1)$, ki je kot ploskev stopnje $(m,n)$ podana s kontrolnimi točkami:
   $$
   \mathbf{b}_{i,j}^{(3)} = \left(1-\frac{i}{m}  \right) \left(1-\frac{j}{n}  \right) \mathbf{b}_{0,0} 
         + \frac{i}{m} \left(1-\frac{j}{n}  \right) \mathbf{b}_{m,0}
         + \left(1-\frac{i}{m}  \right) \frac{j}{n}  \mathbf{b}_{0,n} 
         + \frac{i}{m} \frac{j}{n} \mathbf{b}_{m,n} 
   $$
\end{enumerate}
Coonsova ploskev $\mathbf{p}$ je definirana s kontrolnimi točkami 
$$\mathbf{b}_{i,j} := \mathbf{b}_{i,j}^{(1)}+\mathbf{b}_{i,j}^{(2)}+\mathbf{b}_{i,j}^{(3)}.$$

% Dodamo primer in sliko 

\section{Lastnosti Coonsovih ploskev}

Coonsova ploskve minimzirajo zasuk, definiran kot:
\begin{equation}
   \label{eq:min}
   \int_{[0,1]^2} \left( \frac{\partial^2}{\partial u \partial v}\mathbf{p}(u,v) \right) dS.
\end{equation}
Torej coonsova ploskve doseže minimum izraza \eqref{eq:min}.

Posledica tega je, da so coonsova ploskve lahko v primerih preveč ravne in ne interpolirajo dobro kontrolnih točk.


\end{document}